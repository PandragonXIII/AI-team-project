\documentclass{article}


% if you need to pass options to natbib, use, e.g.:
%     \PassOptionsToPackage{numbers, compress}{natbib}
% before loading neurips_2023


% ready for submission
\usepackage[preprint]{neurips_2023}


% to compile a preprint version, e.g., for submission to arXiv, add add the
% [preprint] option:
%     \usepackage[preprint]{neurips_2023}


% to compile a camera-ready version, add the [final] option, e.g.:
%     \usepackage[final]{neurips_2023}


% to avoid loading the natbib package, add option nonatbib:
%    \usepackage[nonatbib]{neurips_2023}


\usepackage[utf8]{inputenc} % allow utf-8 input
\usepackage[T1]{fontenc}    % use 8-bit T1 fonts
\usepackage{hyperref}       % hyperlinks
\usepackage{url}            % simple URL typesetting
\usepackage{booktabs}       % professional-quality tables
\usepackage{amsfonts}       % blackboard math symbols
\usepackage{nicefrac}       % compact symbols for 1/2, etc.
\usepackage{microtype}      % microtypography
\usepackage{xcolor}         % colors

\usepackage{CJKutf8}
\usepackage{graphicx}
\usepackage{subfigure}
\usepackage{amsmath}
\usepackage{listings}


\title{基于强化学习和马尔可夫模型的出租车代理}


% The \author macro works with any number of authors. There are two commands
% used to separate the names and addresses of multiple authors: \And and \AND.
%
% Using \And between authors leaves it to LaTeX to determine where to break the
% lines. Using \AND forces a line break at that point. So, if LaTeX puts 3 of 4
% authors names on the first line, and the last on the second line, try using
% \AND instead of \And before the third author name.


\author{
  % David S.~Hippocampus\thanks{Use footnote for providing further information
  %   about author (webpage, alternative address)---\emph{not} for acknowledging
  %   funding agencies.} \\
  % Department of Computer Science\\
  % Cranberry-Lemon University\\
  % Pittsburgh, PA 15213 \\
  % \texttt{hippo@cs.cranberry-lemon.edu} \\
  % examples of more authors
  % \And
  Xiuyuan Qi \\
  \texttt{qixy1@shanghaitech.edu.cn} \\
  \And
  Ziyang Guo \\
  \texttt{guozy@shanghaitech.edu.cn} \\
  % \And
  % Coauthor \\
  % Affiliation \\
  % Address \\
  % \texttt{email} \\
  % \And
  % Coauthor \\
  % Affiliation \\Anonymous 
  % Address \\
  % \texttt{email} \\
}


\begin{document}
\begin{CJK}{UTF8}{gbsn}


\maketitle


\begin{abstract} 
  % The abstract paragraph should be indented \nicefrac{1}{2}~inch (3~picas) on
  % both the left- and right-hand margins. Use 10~point type, with a vertical
  % spacing (leading) of 11~points.  The word \textbf{Abstract} must be centered,
  % bold, and in point size 12. Two line spaces precede the abstract. The abstract
  % must be limited to one paragraph.
  本项目为使用人工智能完成一个出租车游戏,着重研究了三种出租车代理程序:Search Agent, Reinforcement Agent 和 Markov-Search Agent, 
  并对它们的不同表现进行比较。
  Search Agent采用广度优先搜索算法;Reinforcement Agent采用强化学习算法;Markov-Search Agent 使用隐马尔科夫模型进行决策。
  结果说明,Search Agent 一定得到最优解;在一定的训练之后Reinforcement Agent也可以得到最优解;而在加入了迷雾和天气系统的游戏中,Markov-Search Agent 有着不错的表现。
\end{abstract}


\section{问题描述}
在一个$M\times N$ 的地图中,每个格子与相邻的四个格子连通,如果相邻的两个格子之间存在墙壁,则两者不再连通。

\begin{figure}[htbp]
  \centering
  \includegraphics[width=6cm,height=4cm]{images/map.png}
  \caption{A possible map.}
\end{figure}
每一轮有1名乘客和1个乘客想前往的目的地。

​地图中一共有4个乘车点,每局游戏开始时乘客和目的地会随机刷新在不同的乘车点。出租车会随机出生在地图中。 

出租车每步可以进行上/下/左/右移动(从一个格子移动到与之连通的另一个格子),或者进行载客/下客操作。 

当出租车位于乘客所在的乘车点,且乘客不在车内时,进行上车操作会使乘客转移到车中。否则操作无效。 

​当出租车位于目的地,且乘客在车内时,进行下车操作会使乘客转移到车所在的地块,并且本局游戏结束。否则操作无效。 

\paragraph{\textbf{得分规则}}
\begin{itemize}
\item 无效的上下车:-10 (与乘客不在同一个格子的情况下上车、车上无乘客或不在目的地时下车) 

\item 将乘客送达目的地:+20(乘客在车中、车在目的地时进行下车操作) 

\item 其他:-1 (移动、合法上下车) 
\end{itemize}
Agent 的目的是使得分最大化,即以尽量少的步数将乘客送达目的地。 

若出租车未能在200步内将乘客送达目的地,则本局游戏将强制结束,以最终得分为本局得分。

\subsection{附加规则}
为了增加难度与不确定性、以及添加前后局之间的关联性,对于部分游戏局,我们添加了以下的附加规则:

\subsubsection{迷雾}
在添加了迷雾的游戏中,存在一个额外的参数$V$ ,只有当乘客与出租车的横、纵距离均小于$V$时,出租车才能收到乘客的位置信息,否则出租车无法知道乘客的位置。 

\subsubsection{天气} 
在连续的多局游戏中,存在一个参数“天气”$W$。
共有3种天气:晴天$(W=0)$、阴天$(W=1)$、雨天$(W=2)$。

每局天气固定。
本局的天气决定下一局各天气出现的概率,
同时决定本局中乘客在各乘车点的出现概率。

暂定天气转移矩阵$T$如下:
$$T=P(W_{t+1}|W_t)=\begin{pmatrix}
  0.7 & 0.2 & 0.1 \\
  0.15 & 0.4 & 0.45 \\
  0.3 & 0.4 & 0.3
\end{pmatrix}$$
其中$W_t$为第$t$局时的天气变量。设$W_t$的取值为$w_t$,则$(T)_{w_t,w_{t+1}}=P(w_{t+1}|w_t)$,即$T$中第$w_t$行第$w_{t+1}$列元素为本局天气为$w_t$的情况下下一局天气为$w_{t+1}$的概率。

暂定天气影响乘客概率的矩阵如下:
$$P(L_t|W_t)=\begin{pmatrix}
  0.2 & 0.1 & 0.1 & 0.6 \\
  0.1&0.4&0.4&0.1\\
  0.7&0.1&0.1&0.1
\end{pmatrix}$$
其中$L_t$为第$t$局中的乘客所在乘车点变量,即乘客出现在了4个乘车点中的哪一个。

暂定初始天气分布$P(W_0)$如下:
$$P(W_0)=\begin{pmatrix}
  \frac{1}{3} & \frac{1}{3} & \frac{1}{3}
\end{pmatrix}$$
即第0局中3种天气的出现概率均等。

%\section{模型与表现}
\section{使用的模型}
本项目着重研究了三种游戏 Agent 模型:Search Agent, Reinforcement Agent 和 Markov-Search Agent.

\subsection{Search Agent}
Search Agent 在拥有完整信息的地图中活动(Agent 得知乘客与目的地的确切位置)。

Search Agent 采用广度优先搜索算法 (BFS) ,先搜索一条通往乘客位置的路径,沿路径前往上车点进行上车操作后,再搜索一条通往目的地的路径,前往目的地让乘客下车。

\subsection{Reinforcement Agent}
同 Search Agent 一样, Reinforcement Agent 也得知乘客与目的地位置。

Reinforcement Agent 采用了Q-Learning算法,其中与课堂中例子的区别是存在多个结束状态,
算法上采用observation state-action pair作为Q table的索引。相比可以画在地图上的Q Table
,这样会形成一个多维的Q Table,更类似吃豆人的例子。 

其次,采用exploration function的方式来鼓励探索。更新Q value的公式如下: 
$$
Q(s,a) = \alpha R(s,a,s')+\gamma max_{a'}Q(s',a') + \frac{explore\_constant}{N(s,a)}
$$
其中, $N(s,a)$ 对应在 $s$状态下选择$a$行动的次数;explore\_constant 为超参数,默认为1。

另外,我们在编写对应的函数时还考虑了训练时间的问题:为了避免在学习次数或其他参数改变
时需要较多时间重新训练模型,函数中支持传入已经

\subsection{Markov-Search Agent}
在附加规则下,天气系统可看作是一个隐马尔科夫模型 (HMM),能以贝叶斯网络的形式表示为下图:

\begin{figure}[htbp]
  \centering
  \includegraphics[width=9cm,height=3cm]{images/1.png}
  \caption{Weather HMM}
\end{figure}
Markov-Search Agent 可基于已知信息,计算出当前游戏局乘客在各乘车点的出现概率,据此决定出租车的行动。具体算法如下:
\begin{enumerate}
  \item 使用 Filtering Algorithm 计算当前各天气的概率:
$$P(W_t|l_{0:t})=\alpha P(l_t|W_t)*(P(W_{t-1}|l_{0:t-1})P(W_t|W_{t-1}))$$
其中$*$号表示矩阵对应元素相乘,不显示符号的乘法表示矩阵乘法。$\alpha$为归一化系数,使得结果向量中的各元素和为1. 

$l_t$表示$L_t$的取值。$l_{0:t}$等同于$l_0,l_1,\dots,l_t$.
  \item 计算当前乘客出现在各乘车点的概率:
$$P(L_t|l_{0:t-1})=P(W_{t-1}|l_{0:t-1})P(W_t|W_{t-1})P(L_t|W_t)$$
  \item 根据概率计算每个位置的期望得分
  \item 搜索一条通往期望最高的乘车点的路径;如果到了附近看到没有乘客,则搜索通往期望第二高的乘车点的路径,如此以往
\end{enumerate}

\section{表现评估}
\subsection{Search Agent}
由于 BFS 算法性质保证了搜索结果为最短路径,且载客、送客两步均不可跳过,可知 Search Agent 必然能得到最优解。因此 Search Agent 可作为另外两种 Agent 的表现参考。

\subsection{Reinforcement Agent}
\subsubsection{学习过程}
在本游戏中,如果agent成功将乘客送达目的地,其得分一般为正数。
并且由于存在200次的时限,agent的得分区间为[-2000,20)。在训练过程中可以发现,一个常见的失败情况是
agent一直向着某个方向移动,并在-200分时退出本局。将不同学习率下agent训练不同次数的得分情况
可视化后得到下图。 
\begin{figure}[htbp]
  \centering
  \includegraphics[width=4.8cm,height=6.4cm]{images/small_map_origin.png}
  \caption{scores under different learning rate and training times.}
\end{figure} 

从图中可以看出:随着学习率下降,达到正得分并收敛所需的训练次数逐渐减少。
然而由于负数部分绝对值较大,较难判断完成游戏时采取的策略的好坏(即正分数的相对大小)
因此采用
$$
f(x) = \left\{
\begin{matrix}
0 & x\leq 0 \\
x & x>0\\ 
\end{matrix}
\right.
$$ 

函数来对得分进行处理,结果如下图左。收敛后的得分基本在8分左右,观察学习率为0.4的损失曲线
得知在约1600次训练后loss基本稳定为零,即收敛到了最优解。
\begin{figure}[htbp]
  \centering
  \subfigure[scores under relu function.]{
  \includegraphics[scale = 0.15]{images/full_small_relu.png} \label{ab}
  }
  \subfigure[loss figure with error]{
  \includegraphics[scale = 0.5]{images/loss_bar.png} \label{cd}
  }
  \caption{scores and loss}
\end{figure} 

\subsubsection{表现对比}
在本游戏中,由于Search Agent一定会得到最优解,因此损失曲线同时也是其他Agent
与Search Agent的得分差距。学习率>0.2 时Reinforcement Agent 基本都可以在2000次迭代后收敛
并达到与Search Agent相同的最优效果。
\begin{figure}[htbp]
  \centering
  \includegraphics[scale = 0.2]{images/comp_rein_search.png}
  \caption{scores under different learning rate and training times.}
\end{figure} 

另外,在未收敛的情况下,很大一部分得分为-200;也有少部分情况有更低分。随着训练次数增加
得分一般会呈现先快后慢的增长趋势。 

此外,我们还对游戏环境进行了一定的改动,来测试在更大的地图下该AI的表现。
大地图的字符串如下
\begin{lstlisting}[language = C]
"+-----------------------+",
"| |R: : : : | : : | : : |",
"| | : | : : | : :G| : : |",
"| : : | | : : : : | : : |",
"| : : | | | : : : | : : |",
"| : : : | | | : : |B| : |",
"| : : : | : | : : : | | |",
"| : : :Y| : | : : : : : |",
"+-----------------------+",
\end{lstlisting}
其中,| 代表墙壁; : 代表可以通过; R,G,B,Y代表四个上下车点。在经过测试后发现
其学习过程的得分曲线与小地图中基本一致:都是类似反比函数的形式;
且随学习率提高收敛所需训练次数逐渐降低,并在0.5左右达到最低值。
\begin{figure}[htbp]
  \centering
  \includegraphics[scale = 0.1]{images/big_map_relu.png}
  \caption{scores on big map after ReLU.}
\end{figure} 


\subsection{Markov-Search Agent 在附加规则下的表现} 
在添加了迷雾以及前后连续的天气的游戏中,我们对Markov-Search Agent进行了测试。
得到如下结果: 

\begin{table}[htbp]
  \centering
  \begin{tabular}{cc}
    \toprule
    \cmidrule(r){1-2}
    Average Score  & Standard Division \\
    \midrule
    5.72233     & 6.4862  \\
    \bottomrule
  \end{tabular}
\end{table} 

其得分大概是无迷雾条件下最优得分的72\%,并且标准差也有一定程度的增大,
不过因为存在较大的不确定性,这种差距也是可以理解的。

\section{主要贡献}
郭子杨:Search Agent;天气系统;Markov-Search Agent 

齐修远:基础环境配置;迷雾系统;强化学习Agent;表现可视化 


项目GitHub仓库:\url{https://github.com/PandragonXIII/AI-team-project} 

\section{参考文献}
Gymnasium API: \url{https://gymnasium.farama.org/environments/toy_text/taxi/} 

CS181 Slides


%%%%%%%%%%%%%%%%%%%%%%%%%%%%%%%%%%%%%%%%%%%%%%%%%%%%%%%%%%%%


\end{CJK}
\end{document}